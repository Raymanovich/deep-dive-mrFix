%----------------------------------------------------------------------------------------
%	FONTS AND INPUT ENCODING
%----------------------------------------------------------------------------------------
\usepackage[utf8]{inputenc} % Required for inputting international characters
\usepackage[T1]{fontenc} % Use 8-bit encoding
\usepackage{fourier} % Use the Adobe Utopia font for the document

%----------------------------------------------------------------------------------------
%	PACKAGES AND OTHER DOCUMENT CONFIGURATIONS
%----------------------------------------------------------------------------------------
% \usepackage{amsmath, amsfonts, amsthm} % Math packages
% \usepackage{listings} % Code listings, with syntax highlighting
\usepackage[dutch]{babel} % dutch language hyphenation

\usepackage[backend=biber,style=numeric,sorting=none]{biblatex}
\DeclareLanguageMapping{dutch}{dutch-apa}
\addbibresource{references.bib}

% \usepackage{natbib}

\usepackage{wrapfig} % Wrap text around figures
\usepackage{booktabs} % Required for prettier tables

% \numberwithin{equation}{section} % Number equations within sections (i.e. 1.1, 1.2, 2.1, 2.2 instead of 1, 2, 3, 4)
% \numberwithin{figure}{section} % Number figures within sections (i.e. 1.1, 1.2, 2.1, 2.2 instead of 1, 2, 3, 4)
% \numberwithin{table}{section} % Number tables within sections (i.e. 1.1, 1.2, 2.1, 2.2 instead of 1, 2, 3, 4)
% none of the above seem to actually work

% \setlength\parindent{0pt} % Removes all indentation from paragraphs
% \setlength\parskip{1em plus 0.1em minus 0.2em} % Put a blank line between paragraphs
% Kinda messes with the toc layout and with headings

\widowpenalties 1 10000 % Don't spread a paragraph across two pages but start it on the second page instead (no bastards) 
\raggedbottom

\usepackage{enumitem} % Required for list customisation
\setlist{noitemsep} % No spacing between list items

\usepackage{csquotes}

\usepackage{setspace}

\newcommand{\comment}[1]{} %define a new command for commenting out chunks of latex code

%----------------------------------------------------------------------------------------
%	GRAPHICS
%----------------------------------------------------------------------------------------
\usepackage{float} % makes stuff float
\usepackage{adjustbox} % lets us put stuff in a box and then manipulate the box
\usepackage{graphicx} % Required for inserting images
\graphicspath{{figures/}{./}} % Specifies where to look for included images (trailing slash required)

% \usepackage{pgf-pie} % Package for making pie charts
% \usepackage{pdfpages} % Accomplished just as well with the graphicx package. 


%----------------------------------------------------------------------------------------
%	DOCUMENT MARGINS
%----------------------------------------------------------------------------------------
\usepackage{geometry} % Required for adjusting page dimensions and margins
\geometry{
	paper=a4paper, % Paper size, change to letterpaper for US letter size
	top=2.5cm, % Top margin
	bottom=3cm, % Bottom margin
	left=3cm, % Left margin
	right=3cm, % Right margin
	headheight=0.75cm, % Header height
	footskip=1.5cm, % Space from the bottom margin to the baseline of the footer
	headsep=0.75cm, % Space from the top margin to the baseline of the header
	%showframe, % Uncomment to show how the type block is set on the page
}

%----------------------------------------------------------------------------------------
%	SECTION TITLES
%----------------------------------------------------------------------------------------
% \usepackage{sectsty} % Allows customising section commands

% % \chaptertitlefont{\centering\normalfont\scshape\bfseries}
% \allsectionsfont{\sffamily}
% \sectionfont{\centering\normalfont\scshape\bfseries} % \section{} styling
% \subsectionfont{\sffamily\nohang\raggedright } % \subsection{} styling
% \subsubsectionfont{\normalfont\itshape\nohang\raggedright} % \subsubsection{} styling
% % \paragraphfont{\normalfont\scshape} % \paragraph{} styling
% % \subparagraphfont{\hspace*{5em}}

%----------------------------------------------------------------------------------------
%	HEADERS AND FOOTERS
%----------------------------------------------------------------------------------------
% \usepackage{scrlayer-scrpage} % Required for customising headers and footers

% \ohead*{} % Right header
% \ihead*{} % Left header
% \chead*{} % Centre header
% 
% \ofoot*{} % Right footer
% \ifoot*{} % Left footer
% \cfoot*{\pagemark} % Centre footer

%----------------------------------------------------------------------------------------
%	SUBFILES
%----------------------------------------------------------------------------------------
% \usepackage{subfiles}

%----------------------------------------------------------------------------------------
%	URLS
%----------------------------------------------------------------------------------------
% \usepackage{url}
\usepackage{xurl} % Tells latex how to line-break a url 
\usepackage{hyperref} % Allows for clickable and configurable urls

\hypersetup{
    colorlinks=true,
    linkcolor=black,
    citecolor=black,
    menucolor=black,
    urlcolor=cyan
    }
\urlstyle{same}


